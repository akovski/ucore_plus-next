\documentclass{article}[11pt]

\usepackage{amsmath}
\usepackage{amssymb}
\usepackage{verbatim}
\usepackage{epsf}
\usepackage{graphicx}
\usepackage{hyperref}

\def\colfigsize{\epsfxsize=5in}

\pdfpagewidth 8.5in
\pdfpageheight 11.0in

\setlength\topmargin{0in}
\setlength\evensidemargin{0in}
\setlength\oddsidemargin{0in}
\setlength\textheight{8.0in}
\setlength\textwidth{6.5in}

\title{Building a bootable guest image for Palacios and Kitten}

\begin{document}
\maketitle

\section{Getting the guest image build tools}

In order to build the bootable guest ISO image, we need to build a Linux kernel
 from source and an initial ramdisk file system containing a set of useful
tools. We will use a new directory for demonstration; the root directory for the
following examples is ``\verb+test/+":

\begin{verbatim}
[jdoe@newskysaw ~]$ mkdir test/
\end{verbatim}

\noindent
There are a set of tools and sources that are useful for the guest image
building procedure. You can obtain these resources from our git repositories.
Change to the ``\verb+test/+" directory and clone the resources:

\begin{verbatim}
[jdoe@newskysaw test]$ git clone http://hornet.cs.northwestern.edu:9005/busybox
[jdoe@newskysaw test]$ git clone http://hornet.cs.northwestern.edu:9005/initrd
[jdoe@newskysaw test]$ git clone http://hornet.cs.northwestern.edu:9005/linux-2.6.30.y
\end{verbatim}

\section{Building the ramdisk filesystem}
The guest requires an initial ramdisk filesystem. Jack has made one that you can
leverage; it is temporarily located in his home directory.  You will need sudo
or root access to create the device files when you unpack the archive:

\begin{verbatim}
[jdoe@newskysaw test]$ cp /home/jarusl/initrd/disks/v3vee_initramfs.tar.gz .
[jdoe@newskysaw test]$ sudo tar -C initrd -xzf v3vee_initramfs.tar.gz
\end{verbatim}

\noindent
If you require a custom initial ramdisk filesystem, change to the
``\verb|initrd/initramfs/|" directory and perform the following steps:

\begin{verbatim}
[jdoe@newskysaw initramfs]$ mkdir -p proc sys var/log
\end{verbatim}

\noindent
Edit the ``\verb|init_task|" script and uncomment these lines:

\begin{verbatim}
#mknod /dev/tty0 c 4 0
#mknod /dev/tty1 c 4 1
#mknod /dev/tty2 c 4 2
\end{verbatim}

\pagebreak

\noindent
Create the ``\verb|console|" device. If you have sudo or root access it is
possible to create this device manually:

\begin{verbatim}
[jdoe@newskysaw initramfs]$ sudo mknod dev/console c 5 1
[jdoe@newskysaw initramfs]$ sudo chmod 0600 dev/console
\end{verbatim}

\noindent
If you do not have sudo or root access it is still possible to create the
``\verb|console|" device indirectly through the kernel build.  Change to the
``\verb|initrd/|" directory and create a file called ``\verb|root_files|". Add
the following line:

\begin{verbatim}
nod /dev/console 0600 0 0 c 5 1
\end{verbatim}

\noindent
The ``\verb|root_files|" file is used when building the Linux kernel in the
section Configuring and building the Linux kernel. Finally, create any
additional directories and copy any additional files that you need. Your initial
ramdisk filesystem is prepped and ready for installation of the BusyBox tools as
described in the section Configuring and installing BusyBox tools.



\pagebreak
\begin{figure}[ht]
  \begin{center}
    \colfigsize\epsffile{busyboxConf1.eps}
    \caption{BusyBox configuration}
    \label{fig:busyboxcf1}
  \end{center}
\end{figure}

\begin{figure}[ht]
  \begin{center}
    \colfigsize\epsffile{busyboxConf2.eps}
  \end{center}
  \caption{BusyBox configuration}
  \label{fig:busyboxcf2}
\end{figure}

\section{Configuring and installing BusyBox tools}

BusyBox is a software application released as Free software under the GNU GPL
that provides many standard Unix tools. BusyBox combines tiny versions of many
common UNIX utilities into a single, small executable. For more details on
BusyBox visit \url{http://busybox.net}. To configure BusyBox, in the
``\verb+busybox/+" directory, type the following:

\begin{verbatim}
[jdoe@newskysaw busybox]$ make menuconfig
\end{verbatim}

\noindent
or

\begin{verbatim}
[jdoe@newskysaw busybox]$ make xconfig
\end{verbatim}

\noindent
The BusyBox tools will be installed in the guest's initial ramdisk filesystem;
you can add any tools that you need. There are two required configuration
options. In the
``\verb|BusyBox settings->Build Options|" menu check the
``\verb|Build BusyBox as a static binary (no shared libs)|" option, as shown in
figure \ref{fig:busyboxcf1}, and in the
``\verb|BusyBox settings->Installation Options|" menu set the
``\verb|Busybox installation prefix|" to the path of the
``\verb|initrd/initramfs|" directory, as shown in figure \ref{fig:busyboxcf2}.
After you finish configuring BusyBox, save your configuration and quit the
window. Then, to make the BusyBox tools, type the following:

\begin{verbatim}
[jdoe@newskysaw busybox]$ make
\end{verbatim}
Install the tools to the guest's initial ramdisk filesystem directory:
\begin{verbatim}
[jdoe@newskysaw busybox]$ make install
\end{verbatim}

\begin{figure}[ht]
  \begin{center}
    \colfigsize\epsffile{linuxConf.eps}
  \end{center}
  \caption{Linux Kernel configuration}
  \label{fig:linuxcf}
\end{figure}


\section{Configuring and building the Linux kernel}

The following procedure demonstrates how to configure and build a 32-bit Linux
kernel. Change to the ``\verb|linux-2.6.30.y/|" directory. There is a custom
configuration file ``\verb|jrl-default-config|" which is configured with minimal
kernel options (all unnecessary options are removed to keep the guest booting
process fast). If you are using the custom configuration file type the
following:

\begin{verbatim}
[jdoe@newskysaw linux-2.6.30.y]$ cp jrl-default-config .config
\end{verbatim}

\noindent
Configure the kernel to meet your requirements. For more on configuring and
building Linux kernels, check online. Type the following:

\begin{verbatim}
[jdoe@newskysaw linux-2.6.30.y]$ make ARCH=i386 menuconfig
\end{verbatim}

\noindent
or

\begin{verbatim}
[jdoe@newskysaw linux-2.6.30.y]$ make ARCH=i386 xconfig
\end{verbatim}

\noindent
The kernel must be configured with the initial ramdisk file system directory
(e.g. ``\verb|initrd/initramfs/|"): in the ``\verb|General setup|" menu under
option
``\verb|Initial RAM filesystem and RAM disk support|" set the
``\verb|Initramfs source file(s)|" option to the path of the
``\verb|initrd/initramfs/|" directory, as shown in figure \ref{fig:linuxcf}.
Additionally, if you are using the ``\verb|root_files|" file to create devices
files, add the ``\verb|root_files|" file path, separated by a space, after the
initial ramdisk filesystem directory. When you are finished configuring the
kernel, save your configuration, and build a bootable ISO image:

\begin{verbatim}
[jdoe@newskysaw linux-2.6.30.y]$ make ARCH=i386 isoimage
\end{verbatim}

\noindent
The ISO image can be found here: ``\verb|arch/x86/boot/image.iso|", and will be
used in the section Configuring and building the guest image.


\section{Configuring and building the guest image}

Checkout the updated Palacios repository to the ``\verb|palacios/|" directory.
(You can find instructions for checking out the Palacios repository at
\url{http://www.v3vee.org/palacios/}). The guest creator utility is required for
building the guest image. Change to the ``\verb|palacios/utils/guest_creator|"
directory and build the guest creator utility:

\begin{verbatim}
[jdoe@newskysaw guest_creator]$ make
\end{verbatim}

\noindent
You will get the ``\verb|build_vm|" utility:
\begin{verbatim}
[jdoe@newskysaw guest_creator]$ file build_vm
build_vm: ELF 64-bit LSB executable, x86-64, version 1 (SYSV), dynamically linked
(uses shared libs), for GNU/Linux 2.6.9, not stripped
\end{verbatim}

\noindent
The guest configuration file is written in XML. A sample configuration file is
provided: ``\verb|default.xml|". Make a copy of the default configuration file
named ``\verb|myconfig.xml|" and edit the configuration elements that you are
interested in (if a device is included in the guest configuration file, it
must be configured in the section Configuring and building Palacios or the guest
will not boot). Of particular importance is the ``\verb|files|" element. Comment
out this attribute:

\begin{verbatim}
<file id="boot-cd" filename="/home/jarusl/image.iso" />
\end{verbatim}

\noindent
Add an attribute that specifies the location of the Linux ISO image:

\begin{verbatim}
<file id="boot-cd" filename="../../../linux-2.6.30.y/arch/x86/boot/image.iso" />
\end{verbatim}

\noindent
When you are finished editing the guest configuration save the configuration
file. The guest image consists of the guest configuration file and the Linux
ISO image. Build the guest image with the guest creator utility:

\begin{verbatim}
[jdoe@newskysaw guest_creator]$ ./build_vm myconfig.xml -o guest.iso
\end{verbatim}

\noindent
The guest image, ``\verb+guest.iso+", is embedded in Kitten's
``\verb|init_task|" in the section Configuring and building Kitten.


\pagebreak
\begin{figure}[h]
  \begin{center}
    \colfigsize\epsffile{kittenConf1.eps}
  \end{center}
  \caption{Kitten configuration}
  \label{fig:kittencf}
\end{figure}

\begin{figure}[ht]
  \begin{center}
    \colfigsize\epsffile{kittenConf2.eps}
  \end{center}
  \caption{Kitten configuration}
  \label{fig:kittencf2}
\end{figure}


\section{Configuring and building Palacios and Kitten}
\subsection*{Configuring and building Palacios}

You can find the detailed manual of getting and building Palacios and Kitten 
from scratch in the Palacios website (\url{http://www.v3vee.org/palacios}). Here
we only give the specific requirements related to the procedure of booting the
guest. To configure Palacios, change to the ``\verb|test/palacios/|" directory
and type the following:

\begin{verbatim}
[jdoe@newskysaw palacios]$ make menuconfig
\end{verbatim}

\noindent
or

\begin{verbatim}
[jdoe@newskysaw palacios]$ make xconfig
\end{verbatim}

\noindent
Don't forget to include the devices that your guest image requires. When you
have configured the components you want to build into Palacios, save the
configuration and close the window. To build Palacios type the following:

\begin{verbatim}
[jdoe@newskysaw palacios]$ make
\end{verbatim}
or
\begin{verbatim}
[jdoe@newskysaw palacios]$ make all
\end{verbatim}

\noindent
Once the Palacios static library has been built you can find the library file,
``\verb+libv3vee.a+", in the Palacios root directory.

\subsection*{Configuring and building Kitten}

Configure Kitten. Change to the ``\verb+test/kitten/+" directory and type the
following:

\begin{verbatim}
[jdoe@newskysaw kitten]$ make menuconfig
\end{verbatim}

\noindent
or

\begin{verbatim}
[jdoe@newskysaw kitten]$ make xconfig
\end{verbatim}

\noindent
Under the ``\verb|Virtualization|" menu select the
``\verb|Include Palacios virtual machine monitor|" option. Set the
``\verb|Path to pre-built Palacios tree|" option to the Palacios build tree
path, ``\verb|..\palacios|", as shown in figure \ref{fig:kittencf}. Set the
``\verb|Path to guest OS ISO image|" option to the guest image path,\\
''\verb|../palacios/utils/guest_creator/guest.iso|'', as shown in figure
\ref{fig:kittencf2}.  When you have finished configuring Kitten, save the
configuration and close the window. To build Kitten type the following:

\begin{verbatim}
[jdoe@newskysaw kitten]$ make isoimage
\end{verbatim}

\noindent
This builds the bootable ISO image file with guest image, Palacios, and Kitten.
The ISO file is located in ``\verb+kitten/arch/x86_64/boot/image.iso+".

\pagebreak
\noindent
You have successfully created an ISO image file that can be booted on a machine.
You can boot the file on Qemu using the following sample command:

\begin{verbatim}
[jdoe@newskysaw test]$ /opt/vmm-tools/qemu/bin/qemu-system-x86_64 \
        -smp 1 \
        -m 2047 \
        -serial file:./serial.out \
        -cdrom kitten/arch/x86_64/boot/image.iso \
        < /dev/null
\end{verbatim}

\noindent
We have finished the entire procedure for building a guest image and booting it
on the Palacios VMM. For more updated details, check the Palacios website
\url{http://www.v3vee.org/palacios} and Kitten website
\url{https://software.sandia.gov/trac/kitten} regularly.

\end{document}